% EDIÇÃO E INCLUSÃO DOS ELEMENTOS PRÉ-TEXTUAIS

{\setlength{\parindent}{0pt}
NUNES, Alexia das Neves; REIS, Eliane Santos dos; ROCHA, Elisangela Leite; GOMES, João Fernando de Mello; BENTO, Michely Santiago; NAPOLEÃO, Verônica Chirichella Felicioni. \textbf{MULTICULTURALISMO: Jogo do acolhimento - Roda de conversa para um ambiente sem violência } Relatório Técnico-Científico. (Pedagogia e Licenciatura em Matemática) - \textbf{Universidade Virtual do Estado de São Paulo}.\\

Tutora: \orientadora \;- São Paulo - SP, 2025

}

%-------------------------------------------
% Inserir resumo
%-------------------------------------------

\setlength{\absparsep}{18pt} % ajustando espaçamento dos paragrafos do resumo


\begin{resumo}
	No Brasil, lidamos com diferentes culturas. No contexto multicultural, o bullying físico pode se intensificar quando há preconceito relacionado à cor da pele, nacionalidade, religião ou aparência. Considerando que se trata de crianças do ensino fundamental I em fase de formação de valores e desenvolvimento emocional, foi essencial para o grupo propor ações que promovam o respeito às diferenças, à diversidade cultural e ao próximo. Além disso, buscamos implementar estratégias de prevenção e intervenção contra o bullying, incentivando valores como o respeito, a empatia, o diálogo e a convivência pacífica entre os alunos. 

	\textbf{PALAVRAS-CHAVE}: multiculturalismo, bullying, preconceito, diversidade, respeito as diferenças
\end{resumo}

%-------------------------------------------
% Inserir lista de ilustrações
%-------------------------------------------

\pdfbookmark[0]{\listfigurename}{lof}
\listoffigures*
\clearpage

%-------------------------------------------
% Inserir sumário
%-------------------------------------------

\pdfbookmark[0]{\contentsname}{toc}
\tableofcontents*
\clearpage

% FINAL DOS ELEMENTOS PRÉ TEXTUAIS