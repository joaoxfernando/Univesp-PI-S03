\pagestyle{simple}
\chapter[]{Introdução}
A violência representa uma ameaça ao processo educacional ensino aprendizagem, e isso pode ocasionar consequências a curto e a longo prazo na vida de uma pessoa. A palavra bullying pode ser bem definido como um ato de praticar ou se envolver em violência, seja ela física ou psicológica, de comportamento agressivo intencional e negativo com execução repetida da ação, envolvendo crianças que apresentam relacionamento com desequilíbrio de poder. Em muitos casos o bullying ocorre como forma de apelido, ofensas, agressão, bater, intimidar, zoar. Sendo estas, formas diretas ou indiretas da praticas do bullying. 

No Brasil, adotou-se o termo bullying, que é utilizado na maioria dos países. Sua definição é compreendida como um subconjunto de comportamentos agressivos, sendo de natureza repetitiva e caracterizada por um desequilíbrio de poder. \citeonline{CleoFante} comenta que para alguns pesquisadores são necessários no mínimo três ataques contra a vítima no decorrer do ano para ser configurado como bullying. Na mesma direção o pesquisador \citeonline{DanOlweus}, afirma que o bullying ocorre quando ações violentas são reproduzidas de três a cinco vezes ao ano. 

A agressividade pode estar ligada a diversos fatores, exemplos: transtorno, distúrbio, influência de amigos e familiares, ou seja a violência na escola está ligada ao contexto social, familiar e escolar. \citeonline{BeatrizPereira} descreve o bullying como uma forma de comportamento agressivo, geralmente maldosa, deliberada e com frequência persistente, podendo durar semanas, meses ou anos, sendo as vítimas sempre incapazes de se defender. O autor ainda acrescenta que, três características são básicas para diferenciar o bullying de outros tipos de violência: o mal causado não resulta de provocações; as intimidações são regulares, os agressores em geral são mais fortes e violentos e as vítimas não estão preparadas para se defender. 

Dentro da proposta temática do multiculturalismo, optamos por abordar especificamente o tema do bullying físico. O projeto foi desenvolvido com os alunos do 3º ano do Ensino Fundamental da Escola Estadual Presidente João Goulart (Figura \ref{fig:fachada_escola}), localizada no extremo sul de São Paulo/SP. 

Como recurso pedagógico, utilizamos a literatura infantil, trabalhando com o livro "E se fosse você?", da escritora Sandra Saruê. A obra serviu como ponto de partida para reflexões e atividades sobre empatia, respeito às diferenças e as consequências do bullying no ambiente escolar. 

\begin{figure}
	\centering
	\includegraphics[width=0.75\textwidth]{fachada1.jpg}
	\caption{Fachada da E. E. Prof João Goulart}
	\label{fig:fachada_escola}
\end{figure}

\chapter{Desenvolvimento}
O desenvolvimento deste projeto foi pautado em uma pesquisa de campo, com a realização de um estudo de caso junto aos alunos do 3º ano do Ensino Fundamental da Escola Estadual Presidente João Goulart, situada no extremo sul da cidade de São Paulo/SP. 

A partir da observação direta e da interação com a comunidade escolar e o entorno, foram identificadas situações recorrentes de desrespeito entre os alunos, como o uso de apelidos ofensivos, agressões físicas e verbais, exclusão social e comportamentos indisciplinados. Essas evidências revelam um cenário preocupante, que compromete tanto o ambiente escolar quanto o processo de ensino aprendizagem. 

Diante desse contexto, a equipe de pesquisa propôs o desenvolvimento do projeto “Promovendo o respeito às diferenças e combatendo o bullying na escola”, com o objetivo de fomentar uma cultura de paz, por meio de leituras significativas, atividades reflexivas e diálogos sobre a valorização da diversidade e a construção de uma ética multicultural. 

Com isso, nosso grupo acredita que a proposta pedagógica busca não apenas intervir nas relações interpessoais dos alunos, mas também estimular uma reflexão coletiva sobre as práticas escolares e os desafios da convivência em contextos marcados por desigualdades sociais, preconceitos e discriminações. 

\section{OBJETIVOS}
\textbf{Objetivo Geral}\\
O principal objetivo do projeto será promover a conscientização dos alunos sobre o bullying físico, suas causas e consequências, incentivando atitudes de respeito, empatia e valorização das diferenças no ambiente escolar. 

\textbf{Objetivo Específico}\\
Estimular o diálogo entre os alunos sobre experiências pessoais e coletivas relacionadas ao tema;

Desenvolver a capacidade de se colocar no lugar do outro por meio da leitura e interpretação da obra literária;

Trabalhar valores ligados ao multiculturalismo, como tolerância, inclusão e convivência pacífica;

Utilizar a literatura como ferramenta pedagógica para abordar questões sociais e emocionais relevantes para a faixa etária.

\section{JUSTIFICATIVA E DELIMITAÇÃO DO PROBLEMA}

Esse projeto justifica-se pela necessidade urgente de enfrentar práticas de bullying físico no ambiente escolar, especialmente em instituições públicas localizadas em regiões mais periféricas, onde as vulnerabilidades sociais são mais visíveis, como é o caso da E.E. Pres. João Goulart, situada no extremo sul da cidade de São Paulo. A ideia é abordar o assunto com os estudantes desenvolvendo habilidades socioemocionais para combater essa prática.

A observação direta realizada com os alunos do 3º ano do ensino fundamental evidenciou comportamentos preocupantes como apelidos ofensivos, agressões físicas, exclusão social e atitudes de desrespeito entre os estudantes. Essas condutas afetam diretamente o ambiente escolar, prejudicando não apenas o rendimento acadêmico, mas também o bem-estar social e emocional dos alunos envolvidos.

Diante desse cenário, a escola vê desafiada a ir além do currículo tradicional, adotando uma abordagem mais humanizada, voltada a formação ética e cidadã. Acreditamos que a escola, enquanto espaço de convivência e construção coletiva, pode e deve promover valores como o respeito, empatia, tolerância e inclusão, especialmente por meio de metodologias que favoreçam o diálogo e a escuta ativa.

A utilização da literatura infantil, representada aqui pelo livro “E Se Fosse Com Você, Uma História de Bullying”, da Sandra Saruê, possibilita trabalhar tais valores de forma sensível, acessível e significativa para alunos nessa faixa etária. 


\section{FUNDAMENTAÇÃO TEÓRICA}

A escola é um espaço essencial de formação integral, onde crianças desenvolvem não apenas competências cognitivas, mas também valores, atitudes e habilidades sociais fundamentais para a convivência em sociedade. No entanto, esse ambiente nem sempre é isento de conflitos, sendo o bullying um dos fenômenos mais recorrentes e prejudiciais ao processo de ensino aprendizagem e à saúde emocional dos alunos.

\textbf{Bullying físico no contexto escolar}

O bullying pode ser definido como um comportamento agressivo, intencional e repetitivo, que ocorre dentro de uma relação de desequilíbrio de poder entre as partes envolvidas \citeonline{DanOlweus}. No caso do bullying físico, essas agressões manifestam-se por meio de empurrões, socos, tapas, chutes e outras formas de violência corporal, frequentemente acompanhadas por ofensas verbais e exclusão social.

\begin{citacao}
	Segundo Menezes, o bullying físico é um dos tipos mais visíveis e, ao mesmo tempo, um dos mais ignorados quando naturalizado no cotidiano escolar como "brincadeiras" ou "coisas de criança". No entanto, suas consequências são graves: afetam o rendimento escolar, a autoestima, a saúde mental das vítimas e, muitas vezes, perpetuam um ciclo de violência. \cite[p.123-138]{IzabelMenezes} 
\end{citacao}


\textbf{Educação multicultural e respeito às diferenças}

Para enfrentar esse tipo de violência, é essencial que a escola adote uma abordagem que valorize a diversidade e promova o respeito às diferenças culturais, étnicas, sociais, religiosas e físicas. A educação multicultural tem como objetivo principal construir um ambiente escolar que reconheça e respeite as identidades diversas dos alunos, desconstruindo preconceitos e práticas excludentes.

Segundo \cite{Candau}, a educação multicultural é um caminho para a construção de uma sociedade mais justa, pois permite que as crianças aprendam desde cedo a conviver com o outro de forma ética e solidária. Nesse sentido, combater o bullying também passa por cultivar a empatia, o diálogo e a consciência crítica sobre os efeitos da desigualdade e da intolerância nas relações escolares.

\textbf{A literatura infantil como recurso pedagógico}

A literatura infantil se apresenta como uma ferramenta potente para mediar reflexões sobre temas complexos como o bullying e o respeito à diversidade. Conforme afirma \cite{Coelho}, a literatura tem a capacidade de ampliar o universo simbólico das crianças, despertando sentimentos, promovendo a identificação com personagens e estimulando a reflexão sobre valores humanos.

No caso deste projeto, o livro "E se fosse você?", de Sandra Saruê, foi escolhido por abordar, de maneira sensível e acessível, situações de exclusão, preconceito e bullying, incentivando o leitor a se colocar no lugar do outro. Essa estratégia pedagógica permite trabalhar conceitos como empatia, solidariedade e respeito de forma significativa, adaptada à faixa etária dos alunos envolvidos. 

\textbf{A pedagogia do diálogo e da escuta}

Com base na pedagogia crítica de Paulo \citeonline{PauloFreire}, é fundamental que a educação seja construída a partir do diálogo, da escuta e do respeito às experiências dos alunos. O combate ao bullying não se faz com punições isoladas, mas com processos educativos que envolvam toda a comunidade escolar na construção de uma cultura de paz, justiça e responsabilidade coletiva.

\section{METODOLOGIA}

O presente projeto será desenvolvido com a turma do 3º ano do Ensino Fundamental da E.E. Pres. João Goulart, situada no extremo sul de São Paulo/SP, durante o ano letivo vigente.

A metodologia adotada será de caráter qualitativo, baseada em um estudo de caso, que prevê a observação, a interação e a intervenção pedagógica junto aos alunos. As etapas principais serão:

\textbf{Diagnóstico inicial}

Realização de observações em sala de aula e nos momentos de convivência escolar para identificar comportamentos relacionados ao bullying físico e suas possíveis causas, além de entrevistas e conversas informais com professores e alunos para compreender a dinâmica do grupo.

\textbf{Uso da literatura infantil como recurso pedagógico}

Introdução do livro “E se fosse você?”, de Sandra Saruê, para a leitura compartilhada em roda de conversa. A obra será utilizada para estimular a empatia, a reflexão e o diálogo sobre as consequências do bullying e a importância do respeito às diferenças


\textbf{Atividades lúdicas e reflexivas}

Proposição de atividades como dramatizações, debates, produções artísticas (desenhos, cartazes, textos) e dinâmicas de grupo, com o objetivo de reforçar os conceitos discutidos durante as rodas de conversa e ampliar a compreensão sobre convivência pacífica.

\textbf{Envolvimento da comunidade escolar}

Organização de encontros com professores, e equipe pedagógica para compartilhar os objetivos do projeto, discutir estratégias de prevenção ao bullying e fortalecer uma rede de apoio à promoção do respeito no ambiente escolar.

\textbf{Registro e análise dos processos}

Documentação das atividades, observações e relatos para acompanhar o desenvolvimento dos alunos e o impacto das ações, possibilitando ajustes na metodologia conforme as necessidades identificadas.


\section{RESULTADOS PRELIMINARES: SOLUÇÃO INICIAL}

Diante do diagnóstico realizado junto aos alunos do 3º ano do Ensino Fundamental da Escola Estadual Presidente João Goulart, que evidenciou a presença de comportamentos de bullying físico, o projeto propõe como solução inicial a utilização do livro “E se fosse você?”, de Sandra Saruê, como recurso pedagógico central.

A estratégia consiste em promover rodas de conversa, leituras compartilhadas e atividades lúdicas que incentivem a empatia e o respeito às diferenças, favorecendo a reflexão sobre as consequências das atitudes agressivas e a valorização da diversidade.

Espera-se que, por meio dessas atividades, os alunos desenvolvam maior consciência sobre o impacto do bullying, reconheçam o valor da convivência pacífica e aprimorem suas habilidades sociais para resolver conflitos de maneira não violenta.

Embora ainda não tenha sido implantado, acredita-se que essa abordagem pedagógica, centrada na literatura infantil e no diálogo, será eficaz para reduzir as práticas de bullying físico e promover um ambiente escolar mais acolhedor e seguro.


%\chapter{Resultados}

%\begin{figure}
%	\centering
%	\includegraphics[width=0.65\textwidth]{fachada1.jpg}
%	\caption{Fachada da E. E. Prof João Goulart}
%	\label{fig:fachada_escola2}
%\end{figure}

%\chapter{Considerações Finais}
%
%\lipsum[1-5]